\chapter*{\center \Large  Abstract}

This project explores the application of machine learning techniques—specifically classification and clustering—across three diverse real-world datasets. The primary objective was to understand, analyze, and compare different models in terms of accuracy and interpretability using Python-based tools and frameworks.

Two classification problems were addressed using the Raisin and HTRU2 datasets. For each dataset, both K-Nearest Neighbors (KNN) and Naïve Bayes classifiers were applied incrementally with increasing feature sets. Confusion matrices and accuracy scores were used to evaluate performance at each step. KNN consistently performed better, particularly in the HTRU2 dataset where it achieved a peak accuracy of 98.27\%.

The third task focused on clustering using the Parking Birmingham dataset. After performing bivariate analysis, the K-Means algorithm was employed, and the optimal number of clusters was determined using the Elbow Method. The final model with $K=3$ provided well-separated and meaningful clusters representing different levels of parking occupancy.

All experiments were conducted in a modular and reusable coding environment, with model logic abstracted into external Python scripts. Visualizations, accuracy comparisons, and clustering plots were generated to support the findings.

This project demonstrates the effectiveness of classic machine learning algorithms when applied systematically, offering valuable insights into model behavior, dataset patterns, and real-world implications.

~\\[1cm]
\noindent\textbf{Keywords:} classification, clustering, machine learning, KNN, K-Means

\vfill
\noindent
\textbf{Report's total word count:} 10,680 (approximately)

%\noindent
%\textbf{GitLab Repository Link:} \url{https://gitlab.com/your-username/ids-semester-project} % (replace with actual link)